% In this file you should put the actual content of the blueprint.
% It will be used both by the web and the print version.
% It should *not* include the \begin{document}
%
% If you want to split the blueprint content into several files then
% the current file can be a simple sequence of \input. Otherwise It
% can start with a \section or \chapter for instance.
This Lean project contains solutions to the exercises in the book \emph{Automatic complexity: a computable measure of irregularity} published by de Gruyter, and excerpts from the book itself.

\chapter*{Preface}

	As the 1968 film \emph{2001: A Space Odyssey} gave an enigmatic and scientifically accurate depiction of space flight,
	so Jeffrey O.~Shallit and Ming-Wei Wang's paper \emph{Automatic complexity of strings}~\cite{MR1897300} from 2001
	described what we can call a ``state odyssey'':
	journeys through the states of a finite automaton that held the promise of further deep exploration.

	While Kolmogorov complexity is only defined ``up to an additive constant'', automatic complexity gives concrete values. When I start up the
	Complexity Guessing Game at \url{http://math.hawaii.edu/wordpress/bjoern/software/web/complexity-guessing-game/}
	on November 14, 2021, I am presented with the string $x=011000111001010$ of length 15, and asked to guess its complexity,
	from 1 to 8.
	As we shall see in this book, in particular in \Cref{chap:FMS},
	the best bet is to choose the largest complexity offered (8 in this case) unless you spot something very special about $x$.
	This is correct for our $x$ and next I am asked about the length 25 word
	\begin{equation}\label{y-word}
		y=1011001111101111010100101
	\end{equation}
	and for a guess for its complexity, from 1 to 13.
	Again I choose the maximum, but in this case, the game responds that the complexity is 12
	and that there is a complexity \emph{deficiency} of 1.

	The idea of complexity (or randomness) deficiency comes from the study of Kolmogorov complexity.
	However, automatic complexity is a more manageable (and computable) measure of irregularity.
	When we say ``irregularity'' here it is partly a pun:
	automatic complexity is based on finite automata, that accept regular languages,
	so irregularity indicates a failure of small finite automata to uniquely identify the string in a sense.

	The game provides the finite automaton for $y$ from \Cref{y-word} shown in \Cref{y-automaton}.
	\begin{figure}
		\centering
		\[
			\xymatrix{
									&								& *+[Fo]{q_{\mathrm A}}\ar[dr]_1   &			   &\\
									& *+[Fo]{q_9}\ar[ur]_1			&				   & *+[Fo]{q_{\mathrm B}}\ar[d]_0  \\
									& *+[Fo]{q_8}\ar[u]_1\ar[dl]_0	&				   & *+[Fo]{q_7}\ar[ll]_1			\\
				*+[Fo]{q_3}\ar[r]_1 & *+[Fo]{q_4}\ar[rr]_0			&				   & *+[Fo]{q_5}\ar[r]_0		   & *+[Fo]{q_6}\ar[ul]_1\ar[dl]_1\\
									& *+[Fo]{q_2}\ar[ul]_1\ar[dr]_1	&				   & *+[Fo]{q_1}\ar[ll]_0		   & \\
									&\text{start}\ar[r]				& *+[Foo]{q_0}\ar[ur]_1
			}
\]
\begin{document}
\begin{tikzpicture}
  \draw(0,0)--(6,6);
\end{tikzpicture}
% \[
% \begin{tikzcd}
%     & & q_{\mathrm{A}} \arrow[dr, "1"] & & \\
%     & q_9 \arrow[ur, "1"] & & q_{\mathrm{B}} \arrow[d, "0"] \\
%     & q_8 \arrow[u, "1"] \arrow[dl, "0"] & & q_7 \arrow[ll, "1"] \\
%     q_3 \arrow[r, "1"] & q_4 \arrow[rr, "0"] & & q_5 \arrow[r, "0"] & q_6 \arrow[ul, "1"] \arrow[dl, "1"] \\
%     & q_2 \arrow[ul, "1"] \arrow[dr, "1"] & & q_1 \arrow[ll, "0"] & \\
%     & \text{start} \arrow[r] & q_0 \arrow[ur, "1"]
% \end{tikzcd}
% 		\]
		\caption{An automaton provided by the \emph{Complexity Guessing Game}.}\label{y-automaton}
	\end{figure}
	\paragraph*{Acknowledgments.}
	I am grateful to many people.
	\begin{itemize}
		\item Andrew J.I.~Jones, Dag Normann, Theodore A.~Slaman, and many others mentored me in computability and logic.
		\item In a Discrete Mathematics class in Spring 2009, students
			Jason Axelson, Chris Ho and Aaron Kondo wrote a C program that calculated the complexity of all strings of length at most 7.
			The dedication they put into that program fueled my interest in automatic complexity.
		\item Logan Axon wrote the first Python script for automatic complexity.
		\item Kayleigh K.~Hyde's 2013 Master's project and
			her proof of the sharp upper bound for nondeterministic automatic complexity sparked my interest in proving theorems in this area.
		\item Students who have worked with me on automatic complexity include Samuel D.~Birns, Calvin K.~Bannister, Swarnalakshmi (Janani) Lakshmanan and Daylan K.~Yogi.
		\item Jeff Shallit, Achilles Beros, Nikolai Vereshchagin, Sasha Shen, Andr\'e Nies, Frank Stephan and Angeliki Koutsoukou-Argyraki
			provided encouragement and interesting discussions.
	\end{itemize}

	This research was supported in part by a grant from Decision Research Corporation
	(University of Hawai\textquoteleft i Foundation Account \#129-4770-4).
	This work was partially supported by a grant from the Simons Foundation (\#704836 to Bj\o rn Kjos-Hanssen).

	While I have tried to keep this book \emph{akamai}, errors may occur and are my responsibility.
	I would be grateful to receive reports at \texttt{bjoernkh+acmoi@hawaii.edu}.

\begin{flushright}	
	Honolulu, October 2023
\end{flushright}


\chapter{The incompressibility theorem}\label{chap:FMS}

	Shallit and Wang showed that the automatic complexity $A(x)$ satisfies $A(x)\ge n/13$ for almost all $x\in{\{\mt{0},\mt{1}\}}^n$.
	They also stated that Holger Petersen had informed them that the constant 13 can be reduced to 7.
	Here we show that it can be reduced to $2+\epsilon$ for any $\epsilon>0$.
	The result also applies to nondeterministic automatic complexity $A_N(x)$.
	In that setting the result is tight inasmuch as $A_N(x)\le n/2+1$ for all $x$.

\bibliographystyle{plain}
\bibliography{ac-acmoi}
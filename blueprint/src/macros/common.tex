% In this file you should put all LaTeX macros and settings to be used both by
% the pdf version and the web version.
% This should be most of your macros.

% The theorem-like environments defined below are those that appear by default
% in the dependency graph. See the README of leanblueprint if you need help to
% customize this. 
% The configuration below use the theorem counter for all those environments
% (this is what the [theorem] arguments mean) and never resets it.
% If you want for instance to number them within chapters then you can add
% [chapter] at the end of the next line.
\newtheorem{theorem}{Theorem}
\newtheorem{proposition}[theorem]{Proposition}
\newtheorem{lemma}[theorem]{Lemma}
\newtheorem{corollary}[theorem]{Corollary}

\theoremstyle{definition}
\newtheorem{definition}[theorem]{Definition}

\newcommand{\N}{\ensuremath{\mathbb{N}} }
\newcommand{\R}{\ensuremath{\mathbb{R}} }
\newcommand{\ent}{\mathcal{H}} %entropy, TCS.tex

\newcommand{\eps}{\varepsilon}
\newcommand{\abs}[1]{\lvert#1\rvert}
\newcommand{\ceil}[1]{\left\lceil#1\right\rceil}
\newcommand{\floor}[1]{\left\lfloor#1\right\rfloor}

\newcommand{\st}{\mid} %such that, in set builder notation
\newcommand{\fr}{:} %function from A to B

\newcommand{\concat}{\mathbin{{+}{+}}}
\newcommand{\dolon}{\mathbin{{:}{;}}}
\newcommand{\lolon}{\mathbin{{:}{:}}}

\newcommand{\LPI}{n+1-m-\sum_{i=1}^m (\alpha_i-2) \abs{x_i}\le 2q}
\DeclareMathOperator{\Lookback}{Lookback}
\DeclareMathOperator{\Var}{Var}
\newcommand{\Fib}{\chi}
\newcommand{\pathc}{\pi}
\newcommand{\wordc}{\omega}
\newcommand{\norm}[1]{\|#1\|}

\DeclareMathOperator{\Acc}{Acc}
\DeclareMathOperator{\logAcc}{logAcc}
\DeclareMathOperator{\perm}{perm}
\newcommand*\ci[1]{\langle #1\rangle}
\newcommand*\di[1]{\| #1 \|}
\newcommand{\mt}{\mathtt}
\newcommand{\restrict}{\upharpoonright}

\newcommand{\la}{\langle}
\newcommand{\ra}{\rangle}

\newcommand{\F}{\mathcal F}
\newcommand{\E}{\mathbb E}
% \DeclareMathOperator{\LC}{LC}
% \DeclareMathOperator*{\argmax}{arg\, max}
% \DeclareMathOperator{\logit}{logit}
% \DeclareMathOperator*{\argmin}{arg\, min}
% \renewcommand{\bar}[1]{\(\overline{\text{#1}}\)}
% \newcommand{\ol}{\overline}
% \DeclareMathOperator{\first}{first}
% \DeclareMathOperator{\last}{last}
